\documentclass[11pt]{article}
\usepackage[english]{babel}
\usepackage{natbib}
\usepackage{url}
\usepackage[utf8x]{inputenc}
\usepackage{amsmath}
\usepackage{graphicx}
\graphicspath{{images/}}
\usepackage{parskip}
\usepackage{fancyhdr}
\usepackage{vmargin}
\usepackage{multirow}
\usepackage{listings}
\usepackage[makeroom]{cancel}

\usepackage{opensans} 
\renewcommand{\familydefault}{\sfdefault}


\usepackage[T1]{fontenc}
\usepackage{beramono}
\usepackage[usenames,dvipsnames]{xcolor}
\DeclareUnicodeCharacter{65533}{$\sigma$}

%%
%% Julia definition (c) 2014 Jubobs
%%
\lstdefinelanguage{Julia}%
  {morekeywords={abstract,break,case,catch,const,continue,do,else,elseif,%
      end,export,false,for,function,immutable,import,importall,if,in,%
      macro,module,otherwise,quote,return,switch,true,try,type,typealias,%
      using,while},%
   sensitive=true,%
   alsoother={\$},%
   morecomment=[l]\#,%
   morecomment=[n]{\#=}{=\#},%
   morestring=[s]{"}{"},%
   morestring=[m]{'}{'},%
}[keywords,comments,strings]%

\lstset{%
    language         = Julia,
    basicstyle       = \ttfamily,
    keywordstyle     = \bfseries\color{blue},
    stringstyle      = \color{magenta},
    commentstyle     = \color{ForestGreen},
	showstringspaces = false,
	numbers=left,
    stepnumber=1,
}


\usepackage[normalem]{ulem}
\setmarginsrb{2 cm}{1.5 cm}{2 cm}{1.5 cm}{1 cm}{1.5 cm}{1 cm}{1.5 cm}

\title{First Tutorial Submission}								% Title
\author{Torsten Babl}			% Author
\date{\today}											  % Date

\makeatletter
\let\thetitle\@title
\let\theauthor\@author
\let\thedate\@date
\makeatother

\pagestyle{fancy}
\fancyhf{}
\rhead{\theauthor}
\lhead{\thetitle}
\cfoot{\thepage}

\begin{document}

%%%%%%%%%%%%%%%%%%%%%%%%%%%%%%%%%%%%%%%%%%%%%%%%%%%%%%%%%%%%%%%%%%%%%%%%%%%%%%%%%%%%%%%%%

\begin{titlepage}
	\centering
    \vspace*{0.5 cm}
    \includegraphics[scale = 0.75]{UCT.jpg}\\[1.0 cm]	% University Logo
    \textsc{\LARGE University of Cape Town}\\[2.0 cm]	% University Name
    \textsc{\Large EEE3088F}\\[0.5 cm]				% Course Code
    
	\textsc{\large Design I}\\[0.5 cm]				% Course Name
	\rule{\linewidth}{0.2 mm} \\[0.4 cm]
	{ \huge \bfseries \thetitle}\\
	\rule{\linewidth}{0.2 mm} \\[1.5 cm]
	
	\begin{minipage}{0.4\textwidth}
		\begin{flushleft} \large
			\emph{Author:}\\
			\theauthor
			\end{flushleft}
			\end{minipage}~
			\begin{minipage}{0.4\textwidth}
			\begin{flushright} \large
			\emph{Student Number:} \\
			BBLTOR001									% Your Student Number
		\end{flushright}
	\end{minipage}\\[2 cm]
	
	{\large 15 May 2020}\\[2 cm]
 
	\vfill
	
\end{titlepage}

%%%%%%%%%%%%%%%%%%%%%%%%%%%%%%%%%%%%%%%%%%%%%%%%%%%%%%%%%%%%%%%%%%%%%%%%%%%%%%%%%%%%%%%%%
\section{Question 5.1.}
\textbf{Mean: } $\bar{x} = 2,275$ \\
\textbf{Standard deviation:} $\sigma = 0,4143$\\
71,4\% of the values lie within a standard deviation of the mean.\\
This is very close to the 68\% expected.\\
All of the values lie within three standard deviations of the mean. \\
This is also expected of a Gaussian distribution.

\begin{lstlisting}
	using PyPlot
using Statistics


x = [1.9223, 2.1234, 1.9324, 3.00123, 2.3143, 2.4365, 1.65438, 
    1.8223, 2.1789, 2.6544, 2.98903, 2.5142, 2.3465, 1.96548]

xbar = mean(x)
σ = std(x)
println(x)
println("mean = ", xbar)
println("σ = ", σ)

inside_σ = filter(i -> i > xbar - σ && i < xbar + σ, x)

println(inside_σ)
println(length(inside_σ) / length(x))

inside_3σ = filter(i -> i > xbar - 3*σ && i < xbar + 3*σ, x)

println(inside_3σ)
println(length(inside_3σ) / length(x))
\end{lstlisting}

\section{Question 5.7.}




\end{document}
